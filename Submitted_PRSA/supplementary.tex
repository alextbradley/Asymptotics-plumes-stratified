\documentclass{article}
\usepackage{graphicx}
\usepackage{epstopdf, epsfig}
\usepackage{newtxtext}
\usepackage{newtxmath}
\usepackage{hyperref}
\usepackage{color}
\usepackage{siunitx}
\usepackage[square, numbers]{natbib}
\title{Supplementary Information for `Asymptotic Analysis of Subglacial Plumes in Stratified Environments` by Bradley et al.}
\author{Alexander T Bradley, C. Rosie Williams, \\ Adrian Jenkins, Robert Arthern}
\date{}
\begin{document}
\maketitle
\newcommand{\Pb}{\textit{P}_B}  %\frac{L/c}{ \tau}\frac{S_l - S_u}{2S_l},
\newcommand{\lt}{\delta} %dimensionless thermocline length, lt/l0
\newcommand{\Pt}{\textit{P}_T}
\renewcommand{\in}{\text{in}} %subcript on into and out of pycnocline variables
\newcommand{\out}{\text{out}}
\newcommand{\order}[1]{\mathcal{O}(#1)}

This supplementary information provides further details on the analysis presented in the main text. In particular, we provide details of the behaviour of solutions in region three in the limit $X \to X_c$, and explicitly set out our constructed melt rate parametrizations in special cases discussed in \S4 of the main text.

%%%%%%%%%%%%%%%%%%%%%% Analysis of Region 3 as X \to X_c %%%%%%%%%%%%
\section{Analysis of Region Three in the Limit $X \to X_c$}
In this section, we describe the behaviour of solutions of the leading order equations for region three in the limit $X \to X_c$, where the velocity $U$ approaches zero. Recall that these leading order equations are 
\begin{align}
 (DU)' &= U Z_b', \label{E:Region3:mass} \\
0 &= D \Delta \rho Z_b' - U^2, \label{E:Region3:mom}\\
(DU\Delta \rho)'  &=\kappa U \Delta T  \label{E:Region3:buoyancy}\\
0&= (1  - 2\Pt -  Z_b)Z_b'U- U\Delta T - DU Z_b',\label{E:Region3:thermal}
\end{align}
and that the flux $Q = DU$ evolves according to
\begin{equation}\label{E:Region3:Q_ODE}
%\left(Q'\right)^3 =\kappa \left(Z_b'\right)^4 \left\{\left(1 - Z_b\right)Q - 2\Pt\left(Q - Q_\in\right) -\left[1 - Z_b(X_0)\right]Q_\in\right\} + \frac{U_\out^3}{Z_b'(X_0)}\left(Z_b'\right)^4.
\frac{\left[Q'(X)\right]^3}{\left[Z_b'(X)\right]^4} = \kappa \left\{ \left[1 - Z_b(X) - 2P_T\right] Q(X) - \left[1 - Z_b(X_0) - 2P_T\right]Q_\text{in}\right\} + \frac{U_\text{out}^3}{Z_b'(X_0)}.
\end{equation}
The point $X_c$ satisfies 
\begin{equation}
\left[1 - Z_b(X_c)\right]Q_c - 2\Pt\left(Q_c - Q_\in\right) -\left[1 - Z_b(X_0)\right]Q_\in +  \frac{U_\out^3}{\kappa Z_b'(X_0)} = 0.
\end{equation}
where $Q_c = Q(X_c)$.

Since the velocity $U \to 0$ as $X = X_c$, we must introduce rescaled variables to reflect a change in asymptotic order; we therefore introduce
%\begin{align}
%U &\sim \kappa^{1/3} Z_b'(X_c)^{2/3} Q_c^{1/3}(X_c - X)^{1/3}, & D &\sim \kappa^{-1/3} Z_b'(X_c)^{-2/3} Q_c^{2/3}(X_c - X)^{-1/3},\label{E:Region3:X_to_Xc1}\\
%\Delta \rho &\sim  \kappa Z_b'(X_c) (X_c - X), & \Delta T &\sim -\kappa^{-1/3} Z_b'(X_c)^{1/3} Q_c^{2/3}(X_c - X)^{-1/3}\label{E:Region3:X_to_Xc2}
%\end{align}
\begin{equation}\label{E:Region3:Rescaling}
X = X_c + \varepsilon \tilde{X}, \quad Q = Q_c + \varepsilon^\gamma \tilde{Q}
\end{equation}
where $\varepsilon \ll 1$ is arbitrary, $\tilde{X} =\order{1}$ is negative, $\tilde{Q} = \order{1}$ and $\gamma >0$ is to be determined. Inserting~\eqref{E:Region3:Rescaling} into~\eqref{E:Region3:Q_ODE} gives
\begin{multline}\label{E:Region3:RescaledODE}
\varepsilon^{3(\gamma - 1)}\frac{\left(\tilde{Q'}\right)^3}{\left[Z_b'(X_c)\right]^4}\left[1 + \order{\varepsilon} \right]= -\varepsilon\lambda\tilde{X}Z_b'(X_c)Q_c + \\ \varepsilon^\gamma \left[1 - 2\Pt - Z_b(X_c)\right] \tilde{Q} + \order{\varepsilon^2, \varepsilon^{\gamma + 1}}.
\end{multline}
A dominant balance is obtained in~\eqref{E:Region3:RescaledODE} by taking $\gamma = 4/3$. After setting $\gamma = 4/3$ in~\eqref{E:Region3:RescaledODE}, using $Q' = U Z_b'$ (from~\eqref{E:Region3:mass}) and undoing the rescaling~\eqref{E:Region3:Rescaling}, we find 
\begin{equation}\label{E:Region3:U_asym}
U \sim \kappa^{1/3} Z_b'(X_c)^{2/3} Q_c^{1/3}(X_c - X)^{1/3} \quad \text{as}~X \to X_c^-.
\end{equation}
From~\eqref{E:Region3:Rescaling}, we have $Q \sim Q_c + \order{\varepsilon^{4/3}}$. Combining this with~\eqref{E:Region3:mass} gives
\begin{equation}\label{E:Region3:D_asym}
D \sim \kappa^{-1/3} Z_b'(X_c)^{-2/3} Q_c^{2/3}(X_c - X)^{-1/3} \quad \text{as}~X \to X_c^-.
\end{equation}
A balance in the momentum equation~\eqref{E:Region3:mom} gives
\begin{equation}\label{E:Region3:drho_asym}
\Delta \rho \sim  \kappa Z_b'(X_c) (X_c - X)\quad \text{as}~X \to X_c^-.
\end{equation}
while a balance in the thermal driving equation~\eqref{E:Region3:thermal} requires
\begin{equation}\label{E:Region3:dT_asym}
 \Delta T \sim -\kappa^{-1/3} Z_b'(X_c)^{1/3} Q_c^{2/3}(X_c - X)^{-1/3} \quad \text{as}~X \to X_c^-.
 \end{equation}

%%%%%%%%%%%%%%%%%%%% Melt Rate in Exception Cases %%%%%%%%
\section{Melt Rate Construction in Special Cases}
In this section, we explicity set out our constructed melt rate profiles in the special cases described in \S4 of the main text.

The first special case describes the situation in which $\Delta \rho_{\in} - 2 \Pb Z_b'(X_0) < 0$. Our analysis suggests that, in this case, the plume will intrude into the ambient within the pycnocline. Our constructed melt rate takes a linear interpolation across the pycnocline as in the approximation presented in the main text (equation (4.12) therein), but modified to include zero speed and thermal driving upon exiting:
\begin{equation}
M_{p} = \begin{cases} 
M_{p,1~} \text{~\quad [equation~(4.1)]}  & 0 < X < X_0 - N_l \lt,\\
M_{p,2~} \text{~\quad [equation~(4.4)]} & X_0 - N_l \lt < X <X_{\text{sep}} ,\\
0  &  X > X_{\text{sep}},\\
\end{cases}
\end{equation}
where $X_{\text{sep}}$ is the value of $X$ at which $M_{p,2} = 0$.

These second exception case occurs when no physically relevant solution of~(4.7), which the `cross-over' point $X^*$ must satisfy. In this case,  the scaled melt rate takes values
\begin{equation}\label{E:MeltRate:AllRegions_noroot}
M_{p} = \begin{cases} 
M_{p,1~} \text{~\quad [equation~(4.1)]}  & 0 < X < X_0 - N_l \lt,\\
M_{p,2~} \text{~\quad [equation~(4.4)]} & X_0 - N_l \lt < X < X_0 + N_l \lt,\\
M_{p,3l~} \text{\quad [equation~(4.8)]} &  X > X_0 + N_l \lt.
\end{cases}
\end{equation}.
Finally, if the computed termination point $X_c$ does not satisfy $X_c > X^*$, we take $M_p$ as in~\eqref{E:MeltRate:AllRegions_noroot}.

\end{document}